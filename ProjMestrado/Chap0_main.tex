%% abtex2-modelo-trabalho-academico.tex, v-1.9.7 laurocesar
%% Copyright 2012-2018 by abnTeX2 group at http://www.abntex.net.br/ 
%%
%% This work may be distributed and/or modified under the
%% conditions of the LaTeX Project Public License, either version 1.3
%% of this license or (at your option) any later version.
%% The latest version of this license is in
%%   http://www.latex-project.org/lppl.txt
%% and version 1.3 or later is part of all distributions of LaTeX
%% version 2005/12/01 or later.
%%
%% This work has the LPPL maintenance status `maintained'.
%% 
%% The Current Maintainer of this work is the abnTeX2 team, led
%% by Lauro César Araujo. Further information are available on 
%% http://www.abntex.net.br/
%%
%% This work consists of the files abntex2-modelo-trabalho-academico.tex,
%% abntex2-modelo-include-comandos and abntex2-modelo-references.bib
%%

% ------------------------------------------------------------------------
% ------------------------------------------------------------------------
% abnTeX2: Modelo de Trabalho Academico (tese de doutorado, dissertacao de
% mestrado e trabalhos monograficos em geral) em conformidade com 
% ABNT NBR 14724:2011: Informacao e documentacao - Trabalhos academicos -
% Apresentacao
% ------------------------------------------------------------------------
% ------------------------------------------------------------------------

\documentclass[
% -- opções da classe memoir --
12pt,				% tamanho da fonte
openright,			% capítulos começam em pág ímpar (insere página vazia caso preciso)
twoside,			% para impressão em recto e verso. Oposto a oneside
a4paper,			% tamanho do papel. 
% -- opções da classe abntex2 --
%chapter=TITLE,		% títulos de capítulos convertidos em letras maiúsculas
%section=TITLE,		% títulos de seções convertidos em letras maiúsculas
%subsection=TITLE,	% títulos de subseções convertidos em letras maiúsculas
%subsubsection=TITLE,% títulos de subsubseções convertidos em letras maiúsculas
% -- opções do pacote babel --
english,			% idioma adicional para hifenização
french,				% idioma adicional para hifenização
spanish,			% idioma adicional para hifenização
brazil				% o último idioma é o principal do documento
]{abntex2}

% ---
% Pacotes básicos 
% ---
\usepackage{lmodern}			% Usa a fonte Latin Modern			
\usepackage[T1]{fontenc}		% Selecao de codigos de fonte.
\usepackage[utf8]{inputenc}		% Codificacao do documento (conversão automática dos acentos)
\usepackage{lastpage}			% Usado pela Ficha catalográfica
\usepackage{indentfirst}		% Indenta o primeiro parágrafo de cada seção.

\usepackage{color, colortbl}	% Controle das cores
\usepackage[usenames,
            dvipsnames,
            svgnames,
            table]{xcolor}	    % Controle das cores
\usepackage{graphicx}			% Inclusão de gráficos
\usepackage{microtype} 			% para melhorias de justificação
\usepackage{multirow}           % habillitar multi linhas em tabeas
\usepackage{tabularx}
\usepackage[table]{xcolor}      % permitir colorir linhas/colunas da tabela
\usepackage{hyperref}           % referências com hiperlink
\usepackage{siunitx,booktabs}           % configuraçoes extras de tabela
\usepackage{longtable}          % para construir tabelas que iniciam em uma página e terminam na outra
\usepackage{threeparttablex}     % ambiente para permitir justificar o "caption" ao tamnaho da tabela
\usepackage{verbatim}           % 
\usepackage{mathtools,
            amsmath,
            amsbsy,
            amssymb,
            amsfonts,
            dsfont,
            slashed}   % Pacotes para auxílio em ambientes matemáticos 
\usepackage[font=footnotesize,
            labelfont=bf]
            {subcaption,caption}   % config p os rótulos de tabelas e figuras e habilitar inserção de sublegendas. ex: figuras com múltiplas imagens
\usepackage{upgreek}            % simbolos matemáticos maiúsculos
\usepackage{float}              % obriga ao corpo flutuante ficar na posição que está no código LATEX [H]
\usepackage{lscape}
\usepackage{multicol}           % ambiente para configuração de ambientes multicolunas
%\usepackage{nonfloat}
\usepackage{caption}
\usepackage{UFBA}               % personalizações para a UFBA
\usepackage{eurosym}
\usepackage{textcomp}
\usepackage{gensymb}
\usepackage{etoc}
\usepackage{nomencl}
\usepackage{lineno}             % exibir o número da linha para revisão
\makenomenclature
\usepackage[nohints,tight]{minitoc} 
\setcounter{minitocdepth}{4}
\usepackage{epstopdf}
%\usepackage{fnpct}
%\usepackage{ftnxtra}
%\usepackage{stfloats}
\usepackage[d]{esvect}
\usepackage{forest}
\usetikzlibrary{decorations.pathreplacing,calligraphy,positioning}

% ---
% Pacotes adicionais, usados apenas no âmbito do Modelo Canônico do abnteX2
% ---
\usepackage{lipsum}				% para geração de dummy text
% ---

% ---
% Pacotes de citações
% ---
\usepackage[brazilian,hyperpageref]{backref}	 % Paginas com as citações na bibl
\usepackage[alf]{abntex2cite}	% Citações padrão ABNT

% --- 
% CONFIGURAÇÕES DE PACOTES
% --- 

% ---
% Configurações do pacote backref
% Usado sem a opção hyperpageref de backref
\renewcommand{\backrefpagesname}{Citado na(s) página(s):~}
% Texto padrão antes do número das páginas
\renewcommand{\backref}{}
% Define os textos da citação
\renewcommand*{\backrefalt}[4]{
    \ifcase #1 %
    Nenhuma citação no texto.%
    \or
    Citado na página #2.%
    \else
    Citado #1 vezes nas páginas #2.%
    \fi}%
% ---

% ----------------------------------------------------------
% Informações de dados para CAPA e FOLHA DE ROSTO
% ----------------------------------------------------------
\titulo{Título do Trabalho.}
\subtitulo{}  %% Inserir subtítulo, se houver
\autor{Fulano(a) de Tal}
\local{Salvador}
\data{xxxx}
\orientador{Dr. zzzzzzzzzzzzzzzzz}
%\orientador{Dr. Eduardo F. de Simas Filho -- UFBA}
%\coorientador[Co-advisors:]{Dr. Eduardo F. de Simas Filho -- UFBA \hphantom{Co-advisors: } Dr. Bertrand Laforge -- Sorbonne Université}
\instituicao{Universididade Federal da Bahia -- UFBA}
\faculdade{Departamento de Engenharia Elétrica}
\programa{Programa de Pós-graduação em Engenharia Elétrica -- PPGEE}
\tipotrabalho{Pré-Projeto de Mestrado}

% O preambulo deve conter o tipo do trabalho, o objetivo, 
% o nome da instituição e a área de concentração 
\preambulo{xxxxx xxxxxxxx xxxxxxx xxxxxx xxxxxx xxxxx xxxxx xxxxx xxxxxxxxxx xxxxxxxxx.}
% ---


% ---
% Configurações de aparência do PDF final

% alterando o aspecto da cor azul
\definecolor{blue}{RGB}{41,5,195}

% informações do PDF
\makeatletter
\hypersetup{
    %pagebackref=true,
    pdftitle={\@title}, 
    pdfauthor={\@author},
    pdfsubject={\imprimirpreambulo},
    pdfcreator={LaTeX with abnTeX2},
    pdfkeywords={abnt}{latex}{abntex}{abntex2}{trabalho acadêmico}, 
    colorlinks=true,       		% false: boxed links; true: colored links
    linkcolor=blue,          	% color of internal links
    citecolor=blue,        		% color of links to bibliography
    filecolor=magenta,      		% color of file links
    urlcolor=blue,
    bookmarksdepth=4
}
\makeatother
% --- 

% ---
% Posiciona figuras e tabelas no topo da página quando adicionadas sozinhas
% em um página em branco. Ver https://github.com/abntex/abntex2/issues/170
\makeatletter
\setlength{\@fptop}{5pt} % Set distance from top of page to first float
\makeatother
% ---

% ---
% Possibilita criação de Quadros e Lista de quadros.
% Ver https://github.com/abntex/abntex2/issues/176
%

% --- 
% Espaçamentos entre linhas e parágrafos 
% --- 

% O tamanho do parágrafo é dado por:
\setlength{\parindent}{1.3cm}

% Controle do espaçamento entre um parágrafo e outro:
\setlength{\parskip}{0.2cm}  % tente também \onelineskip

% ---
% compila o indice
% ---
\makeindex
% ---

% ----
% pasta pas as figuras
% \graphicspath{{subdir1/}{subdir2/}{subdir3/}...{subdirn/}}
\graphicspath{{./Figs/}}

% ----
% Início do documento
% ----
\begin{document}
    
    % Seleciona o idioma do documento (conforme pacotes do babel)
    %\selectlanguage{english}
    \selectlanguage{brazil}
    
    % Retira espaço extra obsoleto entre as frases.
    \frenchspacing 
    
    % ----------------------------------------------------------
    % ELEMENTOS PRÉ-TEXTUAIS
    % ----------------------------------------------------------
    % \pretextual
    
    % ---
    % Capa
    % ---
    \imprimircapa
    % ---
    
    % ---
    % Folha de rosto
    % (o * indica que haverá a ficha bibliográfica)
    % ---
    \imprimirfolhaderosto*
    % ---
    
    % ---
    % Inserir a ficha bibliografica
    % ---
    
    % Isto é um exemplo de Ficha Catalográfica, ou ``Dados internacionais de
    % catalogação-na-publicação''. Você pode utilizar este modelo como referência. 
    % Porém, provavelmente a biblioteca da sua universidade lhe fornecerá um PDF
    % com a ficha catalográfica definitiva após a defesa do trabalho. Quando estiver
    % com o documento, salve-o como PDF no diretório do seu projeto e substitua todo
    % o conteúdo de implementação deste arquivo pelo comando abaixo:
    %
    % \begin{fichacatalografica}
    %     \includepdf{fig_ficha_catalografica.pdf}
    % \end{fichacatalografica}
    
    % ---
    % inserir o sumario
    % ---
    \pdfbookmark[0]{\contentsname}{toc}
    \tableofcontents*
    \cleardoublepage
    % ---
    
    
    
    % ----------------------------------------------------------
    % ELEMENTOS TEXTUAIS
    % ----------------------------------------------------------
    \textual
    
% ----------------------------------------------------------
% Introdução - CAPITULO 1
% ----------------------------------------------------------
\chapter[Introdução]{Introdução}
%\addcontentsline{toc}{chapter}{Introdução}
% ----------------------------------------------------------
%\minitoc % summary per chapter
\input{Chap1_Introducao}


% ----------------------------------------------------------
% FCAPITULO 2
% ----------------------------------------------------------
\chapter[Objetivos]{Objetivos}
%\addcontentsline{toc}{chapter}{Pesquisa Bibliográfica}
\minitoc
\section{Objetivos}

\subsection{Geral}

dfdfdfdfdf

\subsection{Específicos}

dfdfdfdfdf

% ----------------------------------------------------------
% CAPITULO 3
% ----------------------------------------------------------
\chapter[Justificativa]{Justificativa}\label{chap:Método}
%\addcontentsline{toc}{chapter}{Pesquisa Bibliográfica}
\minitoc
\input{Chap3_Justificativa}

% ----------------------------------------------------------
% CAPITULO 4
% ----------------------------------------------------------
\chapter[Método]{Método}\label{chap:results}
%\addcontentsline{toc}{chapter}{Resultados}
\minitoc
\input{Chap4_Metodo}


% ----------------------------------------------------------
% CAPITULO 5 
% ----------------------------------------------------------
\chapter[Metas e Cronogramas]{Metas e Cronogramas}\label{chap:Cronograma}
%\addcontentsline{toc}{chapter}{Metas e Cronogramas}
\minitoc
\input{Chap5_MetasCronogramas}

% ----------------------------------------------------------
% CAPITULO 6 
% ----------------------------------------------------------
\chapter[Riscos Potenciais]{Riscos Potenciais}\label{chap:Cronograma}
%\addcontentsline{toc}{chapter}{Metas e Cronogramas}
\minitoc
\input{Chap6_RiscosPotenciais}


% ----------------------------------------------------------
% CAPITULO 7
% ----------------------------------------------------------
\chapter[Transferência de Resultados]{Transferência de Resultados}\label{chap:Cronograma}
%\addcontentsline{toc}{chapter}{Metas e Cronogramas}
\minitoc
\input{Chap7_TransfResultados}

%% ----------------------------------------------------------
%% ELEMENTOS PÓS-TEXTUAIS
%% ----------------------------------------------------------
\postextual
%% ----------------------------------------------------------
%
%% ----------------------------------------------------------
% Referências bibliográficas
%% ----------------------------------------------------------
%%\bibliography{abntex2-modelo-references}

\begin{small}
    %\bibliographystyle{abntex2-alf}
    \bibliography{Chap8_Bibliogaria}
\end{small}

\end{document}