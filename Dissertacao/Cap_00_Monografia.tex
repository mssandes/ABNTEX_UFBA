% ------------------------------------------------------------------------

\documentclass[
	% -- opções da classe memoir --
%	draft,
	11pt,				% tamanho da fonte
	openright,			% capítulos começam em pág ímpar (insere página vazia caso preciso)
	twoside,			% para impressão em verso e anverso. Oposto a oneside
	a4paper,			% tamanho do papel. 
	% -- opções da classe abntex2 --
	%chapter=TITLE,		% títulos de capítulos convertidos em letras maiúsculas
	%section=TITLE,		% títulos de seções convertidos em letras maiúsculas
	%subsection=TITLE,	% títulos de subseções convertidos em letras maiúsculas
	%subsubsection=TITLE,% títulos de subsubseções convertidos em letras maiúsculas
	% -- opções do pacote babel --
	english,			% idioma adicional para hifenização
	french,				% idioma adicional para hifenização
	spanish,			% idioma adicional para hifenização
	brazil				% o último idioma é o principal do documento
	]{abntex2}          % estilo abnt

% ---
% Pacotes básicos 
% ---
\usepackage{lmodern}			% Usa a fonte Latin Modern			
\usepackage[T1]{fontenc}		% Selecao de codigos de fonte.
\usepackage[utf8]{inputenc}		% Codificacao do documento (conversão automática dos acentos)
\usepackage{lastpage}			% Usado pela Ficha catalográfica
\usepackage{indentfirst}		% Indenta o primeiro parágrafo de cada seção.
\usepackage{color, colortbl}	% Controle das cores
\usepackage[usenames,
            dvipsnames,
            svgnames,
            table]{xcolor}	    % Controle das cores
\usepackage{graphicx}			% Inclusão de gráficos
\usepackage{microtype} 			% para melhorias de justificação
\usepackage{multirow}           % habillitar multi linhas em tabeas
\usepackage[table]{xcolor}      % permitir colorir linhas/colunas da tabela
\usepackage{hyperref}           % referências com hiperlink
\usepackage{siunitx,booktabs}           % configuraçoes extras de tabela
\usepackage{longtable}          % para construir tabelas que iniciam em uma página e erminam na outra
\usepackage{threeparttable}     % ambiente para permitir justificar o "caption" ao tamnaho da tabela
\usepackage{verbatim}           % 
\usepackage{mathtools,
            amsmath,
            amsbsy,
            amssymb,
            amsfonts, dsfont}   % Pacotes para auxílio em ambientes matemáticos 
\usepackage[font=footnotesize,
            labelfont=bf]
            {subcaption,caption}   % config p os rótulos de tabelas e figuras e habilitar inserção de sublegendas. ex: figuras com múltiplas imagens
\usepackage{upgreek}            % simbolos matemáticos maiúsculos
\usepackage{float}              % obriga ao corpo flutuante ficar na posição que está no código LATEX [H]
\usepackage{lscape}
\usepackage{multicol}           % ambiente para configuração de ambientes multicolunas
\usepackage{slashbox}           % habiitar divisao diagonal em celulas de tabelas
%\usepackage{nonfloat}
\usepackage{caption}
\usepackage{UFBA}               % personalizações para a UFBA
\usepackage{eurosym}
\usepackage{gensymb}
\usepackage{etoc}
\usepackage[nohints,tight]{minitoc} 
\setcounter{minitocdepth}{4}
%\usepackage{epstopdf}
%\usepackage{fnpct}
%\usepackage{ftnxtra}
%\usepackage{stfloats}
%% ---
%% ---

% ---
%% Catologo de fontes		
%%http://www.tug.dk/FontCatalogue/mathfonts.html
% ---

% ---
% Pacotes adicionais, usados apenas no âmbito do Modelo Canônico do abnteX2
% ---
\usepackage{lipsum}				% para geração de dummy text
% ---

% ---
% Pacotes de citações
% ---
\usepackage[brazilian,hyperpageref]{backref}	 % Paginas com as citações na bibl
\usepackage[alf,bibjustif]{abntex2cite}	% Citações padrão ABNT

% --- 
% CONFIGURAÇÕES DE PACOTES
% --- 

% ---
% Configurações do pacote backref
% Usado sem a opção hyperpageref de backref
\renewcommand{\backrefpagesname}{Citado na(s) página(s):~}
% Texto padrão antes do número das páginas
\renewcommand{\backref}{}
% Define os textos da citação

\renewcommand*{\backrefalt}[4]{
	\ifcase #1 %
	No one citation on the text.%
	\or
	Cited on page #2.%
	\else
	Cited #1 times on pages #2.%
	\fi}%

\newenvironment{figurehere}
  {\par\medskip\noindent\minipage{\linewidth}}
  {\endminipage\par\medskip}

%\newcommand\figurehere[1]{%
%\medskip\noindent\begin{minipage}{\columnwidth}
%\centering%
%#1%
%%figure,caption, and label go here
%\end{minipage}\medskip}

%% Definicao do corpo figurehere quando a figura estiver dentro do ambiente multicols
%\newenvironment{figurehere}
%  {\def\@captype{figure}}
%  {}

\newcommand{\splitcell}[1]{%
  \begin{tabular}{@{}c@{}}\strut#1\strut\end{tabular}%
}

%% Definindo comando para estilo de representacao de vetores
\renewcommand{\vec}[1]{\mathbf{#1}}

%% Definindo espaco entre equacoes nos ambientes eqnarray e equation
\setlength{\jot}{10pt}
% ---

% ----------------------------------------------------------
% Informações de dados para CAPA e FOLHA DE ROSTO
% ----------------------------------------------------------
\titulo{Título do Trabalho}

\subtitulo{Subtítulo se Houver}  %% Inserir subtítulo, se houver

\autor{ALuno Candidato}
\local{Salvador}
\data{2021}
\orientador{Dr.  -- UFBA}
\coorientador{Dr. -- UFBA}
\instituicao{Universidade Federal da Bahia -- UFBA}
\faculdade{Departamento de Engenharia Elétrica}
\programa{Programa de Pós-Graduação em Engenharia Elétrica -- PPGEE}
\tipotrabalho{Exame de qualificação de PhD}

% O preambulo deve conter o tipo do trabalho, o objetivo, 
% o nome da instituição e a área de concentração 
\preambulo{Texto para exame de qualificação de tese de Doutorado apresentada  ao  Programa de Pós-graduação em Engenharia Elétrica da Universidade Federal da Bahia  como um dos requisitos para obtenção do grau de Doutor em Engenharia Elétrica.}
% ---


% ---
% Configurações de aparência do PDF final
% Cores no latex: http://latexcolor.com/

% alterando o aspecto da cor azul
%\definecolor{blue}{RGB}{41,5,195}
% definindo cor usada nos hiperlinks do documento
\definecolor{persianplum}{rgb}{0.44, 0.11, 0.11}
\definecolor{midnightblue}{rgb}{0.1, 0.1, 0.44}
% cor utilizada em celulas de tabelas
\definecolor{lightgray}{rgb}{.9 .9 .9}
\definecolor{gray2}{rgb}{.69 .69 .69}


%\definecolor{orange}{rgb}{0.9, 0.36, 0.03}
% informações do PDF
\makeatletter
\hypersetup{
     	%pagebackref=true,		pdftitle={\@title}, 
		pdfauthor={\@author},
    	pdfsubject={\imprimirpreambulo},
	    pdfcreator={LaTeX with abnTeX2},
		pdfkeywords={abnt}{latex}{abntex}{abntex2}{trabalho acadêmico}, 
		colorlinks=true,       		% false: boxed links; true: colored links
%    	linkcolor=blue,          	% color of internal links
    	citecolor=persianplum,       % color of links to bibliography
    	linkcolor=midnightblue,       % color of internal links
%    	citecolor=black,            % color of links to bibliography
    	filecolor=magenta,      	% color of file links
		urlcolor=blue,
		bookmarksdepth=4
}
\makeatother
% --- 

% --- 
% Espaçamentos entre colunas
% --- 
\setlength{\columnsep}{7mm}

% --- 
% Espaçamentos entre linhas e parágrafos 
% --- 

% O tamanho do parágrafo é dado por:
\setlength{\parindent}{1.3cm}

% Controle do espaçamento entre um parágrafo e outro:
\setlength{\parskip}{0.2cm}  % tente também \onelineskip

% ---
% compila o indice
% ---
\makeindex
% ---

% ----
% pasta para as figuras
% \graphicspath{{subdir1/}{subdir2/}{subdir3/}...{subdirn/}}
%\graphicspath{{/media/mssandes/MSSANDES1/cernbox/Pesquisa/Figuras/}}
\graphicspath{{./Figs/}}
% ----

%%%%%%%%%%%%%%%%%%%%%%%%%%%%%%%%%%%%%%%%%%%%%%%%%%%%%%%%%%%%%%%%%%%%%%%%%%
%%%%%%%%%%%%%%%%%%%%%%%%%%%%%%%%%%%%%%%%%%%%%%%%%%%%%%%%%%%%%%%%%%%%%%%%%
%%%%%%%%%%%%%%%%%%%%%%%%%%%%%%%%%%%%%%%%%%%%%%%%%%%%%%%%%%%%%%%%%%%%%%%%%
% Início do documento
% ----
\begin{document}

% Retira espaço extra obsoleto entre as frases.
\frenchspacing 

% ==========================================================
% ----------------------------------------------------------
% ELEMENTOS PRÉ-TEXTUAIS
% ----------------------------------------------------------
% \pretextual

% ---
% Capa
% ---
\imprimircapa
% ---

% ---
% Folha de rosto
% (o * indica que haverá a ficha bibliográfica)
% ---
\imprimirfolhaderosto*
% ---

% ---
% Inserir a ficha Catalográfica
% ---
% Isto é um exemplo de Ficha Catalográfica, ou ``Dados internacionais de
% catalogação-na-publicação''. Você pode utilizar este modelo como referência. 
% Porém, provavelmente a biblioteca da sua universidade lhe fornecerá um PDF
% com a ficha catalográfica definitiva após a defesa do trabalho. Quando estiver
% com o documento, salve-o como PDF no diretório do seu projeto e substitua todo
% o conteúdo de implementação deste arquivo pelo comando abaixo:
%
% \begin{fichacatalografica}
%     \includepdf{fig_ficha_catalografica.pdf}
% \end{fichacatalografica}

%% Isto é um exemplo de Ficha Catalográfica, ou ``Dados internacionais de
% catalogação-na-publicação''. Você pode utilizar este modelo como referência. 
% Porém, provavelmente a biblioteca da sua universidade lhe fornecerá um PDF
% com a ficha catalográfica definitiva após a defesa do trabalho. Quando estiver
% com o documento, salve-o como PDF no diretório do seu projeto e substitua todo
% o conteúdo de implementação deste arquivo pelo comando abaixo:
%
% \begin{fichacatalografica}
%     \includepdf{fig_ficha_catalografica.pdf}
% \end{fichacatalografica}
\begin{fichacatalografica}
	\vspace*{\fill}					% Posição vertical
	\hrule							% Linha horizontal
	\begin{center}					% Minipage Centralizado
	\begin{minipage}[c]{12.5cm}		% Largura
	
	\imprimirautor
	
	\hspace{0.5cm} \imprimirtitulo  / \imprimirautor. --
	\imprimirlocal, \imprimirdata-
	
	\hspace{0.5cm} \pageref{LastPage} p. : il. (algumas color.) ; 30 cm.\\
	
	\hspace{0.5cm} \imprimirorientadorRotulo~\imprimirorientador\\

%	\hspace{0.5cm} \imprimircoorientadorRotulo~\imprimircoorientador\\
	
	\hspace{0.5cm}
	\parbox[t]{\textwidth}{\imprimirtipotrabalho~--~\imprimirinstituicao,
	\imprimirdata.}\\
	
	\hspace{0.5cm}
%		1. Palavra-chave1.
%		2. Palavra-chave2.
%		I. Orientador.
		1. Redes Neurais Artificias.
		2. Processamento de Sinais.
		3. Reconhecimento de Acordes.
		
		I. Dr. Eduardo F. Simas.

		II. Universidade Federal da Bahia.
		III. Departamento de Engenharia Elétrica.
		IV. Reconhecimento de Acordes Naturais de Guitarra utilizando Redes Neurais Artificiais\\ 			
	
	\hspace{8.75cm} CDU 02:141:005.7\\
	
	\end{minipage}
	\end{center}
	\hrule
\end{fichacatalografica}



% ----------------------------------------------------------
% Inserir errata
% ----------------------------------------------------------
%\begin{errata}
%Elemento opcional da \citeonline[4.2.1.2]{NBR14724:2011}. Exemplo:
%
%\vspace{\onelineskip}
%
%FERRIGNO, C. R. A. \textbf{Tratamento de neoplasias ósseas apendiculares com
%reimplantação de enxerto ósseo autólogo autoclavado associado ao plasma
%rico em plaquetas}: estudo crítico na cirurgia de preservação de membro em
%cães. 2011. 128 f. Tese (Livre-Docência) - Faculdade de Medicina Veterinária e
%Zootecnia, Universidade de São Paulo, São Paulo, 2011.
%
%\begin{table}[htb]
%\center
%\footnotesize
%\begin{tabular}{|p{1.4cm}|p{1cm}|p{3cm}|p{3cm}|}
%  \hline
%   \textbf{Folha} & \textbf{Linha}  & \textbf{Onde se lê}  & \textbf{Leia-se}  \\
%    \hline
%    1 & 10 & auto-conclavo & autoconclavo\\
%   \hline
%\end{tabular}
%\end{table}
%
%\end{errata}
% ---

% ---
% Inserir folha de aprovação
% ----------------------------------------------------------
%% ---
% Inserir folha de aprovação
% ---

% Isto é um exemplo de Folha de aprovação, elemento obrigatório da NBR
% 14724/2011 (seção 4.2.1.3). Você pode utilizar este modelo até a aprovação
% do trabalho. Após isso, substitua todo o conteúdo deste arquivo por uma
% imagem da página assinada pela banca com o comando abaixo:
%
% \includepdf{folhadeaprovacao_final.pdf}
%
\begin{folhadeaprovacao}

  \begin{center}
    \MakeUppercase{\ABNTEXchapterfont\large\imprimirautor}

    \vspace*{\fill}\vspace*{\fill}
    \begin{center}
      \MakeUppercase{\ABNTEXchapterfont\bfseries\Large\imprimirtitulo}

      \MakeUppercase{\ABNTEXchapterfont\bfseries\large\imprimirsubtitulo}
    \end{center}
    \vspace*{\fill}
    
    \hspace{.45\textwidth}
    \begin{minipage}{.5\textwidth}
        \imprimirpreambulo
    \end{minipage}%
    \vspace*{\fill}
   \end{center}
        
   Trabalho aprovado. \imprimirlocal, 16  de dezembro de 2014:

   \assinatura{\textbf{\imprimirorientador} \\ \small{Orientador}}
   \assinatura{\textbf{\imprimircoorientador} \\ \small{Coorientador}}
   \assinatura{\textbf{Convidado} \\ \small{Convidado 3}}
   \assinatura{\textbf{Professor} \\ \small{Convidado 4}}
      
   \begin{center}
    \vspace*{0.5cm}
    {\large\imprimirlocal}
    \par
    {\large\imprimirdata}
    \vspace*{1cm}
  \end{center}
  
\end{folhadeaprovacao}
% ---

% ----------------------------------------------------------

% ---
% Dedicatória
% ----------------------------------------------------------
%\begin{dedicatoria}
   \vspace*{\fill}
   \centering
   \noindent
   \textit{ Dedico esse trabalho aos meus pais,\\
   que são meus grandes incentivadores, e orientadores.} \vspace*{\fill}
\end{dedicatoria}
% ----------------------------------------------------------

% ---
% Agradecimentos
% ----------------------------------------------------------
%\begin{agradecimentos}

Agradeço ao Senhor. O criador, autor e consumador da minha fé por conseguir concluir essa etapa e enxergar a beleza, para mim inquestionável, de sua criação por meio da ciência. À minha família, meus pais, irmão e esposa, pela compreensão e carinho, durante essa jornada pela engenharia. Aos amigos pela torcida, incentivo e orações. Aos meus colegas de curso, e suas contribuições, grande engenheiros que conheci durante a formação. E aos professores que contribuíram significativamente para minha formação acadêmica e profissional. Muito obrigado!

\end{agradecimentos}
% ----------------------------------------------------------

% ----------------------------------------------------------
% Epígrafe
% ----------------------------------------------------------
%\begin{epigrafe}
%    \vspace*{\fill}
%	\begin{flushright}
%		\textit{``Vós sois a nossa carta, \\
%		escrita em nossos corações, \\
%		conhecida e lida por todos os homens''.\\
%		(II Coríntios 2:3)}
%	\end{flushright}
%\end{epigrafe}
% ---

% ----------------------------------------------------------
% RESUMOS
% ----------------------------------------------------------
%\setlength{\absparsep}{18pt} % ajusta o espaçamento dos parágrafos do resumo
\begin{resumo}
 
%% Revisão(?) SIMAS
A pesquisa apresentada tem por objetivo realizar o reconhecimento de acordes naturais, maiores e menores, de uma guitarra elétrica. Para tanto, foi necessário pesquisar sobre o tema rede neural artificial (RNA). Promoveu-se o processo de amostragem dos acordes executados na guitarra para a construção de um banco de dados. No processamento de dados, utilizou-se a  (CZT), para a análise do conteúdo harmônico presente nos acordes, a fim de selecionar quais componentes são relevantes em relação à amplitude. Tais componentes formaram o conjunto de características de cada acorde, que foi apresentado à RNA. O banco de dados construído atendeu os requisitos mínimos do sistema e ao Teorema de Nyquist, para a máxima frequência reproduzida por uma guitarra de 22 trastes, além de garantir que a frequência de amostragem não permitisse a existência de duas componentes harmônicas de interesse entre dois períodos de amostragem. Esse critério foi atendido analisando-se a característica da distribuição das frequências da guitarra. Tal instrumento é de escala temperada – possui o mesmo intervalo entre notas consecutivas. Para cada acorde foi construída a \textit{feature chroma}, vetor cujas 12 componentes indicam o somatório das energias contidas nas bandas de frequência correspondentes a cada uma das 12 notas musicais. Com o auxílio do \textit{nprtool – Network Pattern Recognition} assistente do MATLAB\begin{footnotesize}$^{\textregistered}$\end{footnotesize}, para RNA, com uma única camada oculta, testou-se algumas configurações de rede, objetivando determinar qual configuração seria mais eficiente no que diz respeito ao percentual global. Em todas as configurações testadas o número de neurônios da camada de entrada foi de 12, e para a camada de saída 24, uma saída para cada classe. Por fim, algumas configurações, variando o número de neurônios da camada oculta (4 a 20), resultaram num percentual, global, de acerto entre 84,42\%  e 94,32\%. Os dois melhores resultados foram obtidos para uma configuração contendo 20 neurônios, e a segunda contendo 16 neurônios, 93,88\% e 94,32\%, respectivamente.
 
 

 \textbf{Palavras-chave}: Redes Neurais. Reconhecimento de Acordes. Reconhecimento de Padrões. Recuperação de Informação Musical.
\end{resumo}

% resumo em inglês
\begin{resumo}[Abstract]
 \begin{otherlanguage*}{english}
 

The goal of this research is to promote the recognition of natural chords, major and minor, of an electric guitar. Therefore, it was necessary to research on the topic artificial neural network (ANN). Promoted the sampling process performed on the guitar chords for the construction of a database. In data processing, we used the (CZT), for analyzing the harmonic content present in the chords in order to select which components are relevant for amplitude. These components form the feature set of each chord, which was presented to the ANN. The database built has met the minimum requirements of the system and the Nyquist Theorem, to the maximum frequency reproduced by a 22 fret guitar, and ensure that the sampling rate would not allow the existence of two harmonic components of interest between two periods sampling. This criterion was met by analyzing the characteristics of the distribution of guitar frequencies. This instrument is tempered scale - has the same interval between consecutive notes. For each chord has been constructed feature chroma vector whose components 12 indicate the summation of the energies contained in the frequency bands corresponding to each of 12 musical notes. With the aid of nprtool - Network Pattern Recognition assistant Matlab for ANN with a single hidden layer, we tested some network configurations, in order to determine which configuration would be more efficient as regards the overall percentage. In all tested configurations the number of neurons in the input layer 12, and the output layer 24, one output for each class. Finally, some configurations, varying the number of neurons in the hidden layer (4-20) resulted in a percentage, overall, the arrangement between 84.42\% and 94.32\%. The two best results were obtained for a configuration with 20 neurons, and the second containing 16 neurons, 93.88\% and 94.32\%, respectively.
 
 
   \vspace{\onelineskip}
 
   \noindent 
   \textbf{Keywords}: Neural Networking. Chord Recognition. Patterns Recognition. Music Information Retrieval (MIR).
 \end{otherlanguage*}
\end{resumo}

% ---
% inserir lista de ilustrações
% ---
\pdfbookmark[0]{\listfigurename}{lof}
\listoffigures*
\cleardoublepage
% ---

% ----------------------------------------------------------
% inserir lista de tabelas
% ----------------------------------------------------------
\pdfbookmark[0]{\listtablename}{lot}
\listoftables*
\cleardoublepage
% ---

% ----------------------------------------------------------
% inserir lista de abreviaturas, siglas e símbolos
% ----------------------------------------------------------
%% inserir lista de abreviaturas e siglas
%% ---
\begin{siglas}
  \item[$\mu$C] Microcontrolador
  \item[$\mu$P] Microprocessador
  \item[A\#]   Nota (Acorde) Lá Sustenido  (Maior)
  \item[A\#m]  Acorde Lá Menor
  \item[A]     Nota (Acorde) Lá (Maior)
  \item[Am]    Acorde Lá Menor
  \item[B]     Nota (Acorde) Sí  (Maior)
  \item[B]     Nota (Acorde) Sí (Maior)
  \item[$B_k$] Banda de frequências para a CZT
  \item[Bm]    Acorde Sí Menor
  \item[Bm]    Acorde Sí Menor
  \item[C\#]   Nota (Acorde) Dó Sustenido  (Maior)
  \item[C\#m]  Acorde Dó Menor
  \item[C]     Nota (Acorde) Dó (Maior)
  \item[Cm]    Acorde Dó Menor
  \item[CZT]   \textit{Chirp z-transform}
  \item[$\delta$] Variação em torno da frequência na escala temperada para a CZT
  \item[$\delta_{f}$] Período de amostragem entre dois períodos de amostragem
  \item[$\Delta_{f^T}$] Menor intervalo de frequências para a guitarra utilizada 
  \item[D\#]   Nota (Acorde) Ré Sustenido (Maior)
  \item[D\#m]  Acorde Ré Menor
  \item[D]     Nota (Acorde) Ré (Maior)
  \item[DFT]   \textit{Discret Fourier Transform}
  \item[Dm]    Acorde Ré Menor
  \item[DSP]   \textit{Digital Signal Processing}
  \item[$\mathcal{E}_p$] Energia em cada componente de frequência
  \item[E]     Nota (Acorde) Mi (Maior)
  \item[E]     Nota (Acorde) Mi (Maior)
  \item[Em]    Acorde Mi Menor
  \item[Em]    Acorde Mi Menor
  \item[F\#]   Nota (Acorde) Fá Sustenido (Maior)
  \item[F\#m]  Acorde Fá Menor
  \item[F]     Nota (Acorde) Fá (Maior)
  \item[FFT]   \textit{Fast Fourier Transform}
  \item[$f_s$] Frequência de amostragem
  \item[$f^{Temp}_k$] Frequência da escala temperada
  \item[$f^{FFT}_p$] Componente de frequência dada pela FFT
  \item[Fm]    Acorde Fá Menor
  \item[G\#]   Nota (Acorde) Sol Sustenido (Maior)
  \item[G\#m]  Acorde Sol Menor
  \item[G]     Nota (Acorde) Sol (Maior)
  \item[Gm]    Acorde Sol Menor
%  \item[IA]    Inteligência Artificial
% \item[$l$]
%  \item[$m$]   Número de neurônios para camada oculta
%  \item[$n$]
  \item[N]     Número de pontos para a FFT
  \item[\textit{nprtool}] \textit{Network Pattern Recognition Tool}
%  \item[$p$]
  \item[$\omega{_N}$] Componente de frequência de Fourier
  \item[$\Omega{_k}$] Componente de frequência da escala temperada no domínio da frequência
  \item[RNA]   Rede Neural Artificial
  \item[SVM] \textit{Support Vector Machine}
  \item[TFG]   Trabalho Final de Graduação

\end{siglas}
%% ---

%% ---
%% Inserir lista de Símbolos
%%
%\begin{simbolos}
%  \item[$ \flat $] Bemol
%  \item[$ \# $] Sustenido
%  \item[$ \zeta $] Letra grega minúscula zeta
%  \item[$ \in $] Pertence
%\end{simbolos}
%% ---

% ----------------------------------------------------------
% inserir o sumario
% ----------------------------------------------------------
\pdfbookmark[0]{\contentsname}{toc}
\tableofcontents*
%%%%%%%%%%%%%%%%%%
% insere o sumario do capítulo para auxilio
% no capitulo deve inserir \localtableofcontents*
\etocsettocstyle{}{} 
%%%%%%%%%%%%%%%%%%
\cleardoublepage
% ---

\textual

% ----------------------------------------------------------
% Introdução - CAPITULO 1
% ----------------------------------------------------------
\chapter[Introdução]{Introdução}\label{cap:intro}
%\addcontentsline{toc}{chapter}{Introdução}
% ----------------------------------------------------------

%% ===========================
%%         Introdução
%% ===========================

Esse é um texto exemplo para mostrar como chamar a lista de acrônimos e glossário automáticos. Para treinar uma \gls{rna}. Na primeira aparição no texto usando o comando \verb|\gls{rna}| a sigla é apresentada por extenso. Caso a sigla seja em minúsculo e precise capitalizar, basta usar \verb|\Gls{rna}|.

A lista de constantes é impressa mesmo que não seja invocada. Basta definir os símbolos e/ou constantes na lista de símbolos pelo comando: 

\verb|\newglossaryentry{ec}{type=symbols,name={E$_c$},sort=e,description={Critical Energy}}|

Caso queira chamar um símbolo, a sintaxe é a mesma \verb|\gls{eta}|, \gls{eta}.
%---------------------------------------------------
% Justificativa
%---------------------------------------------------



% ----------------------------------------------------------
% Pesquisa Bibliográfica - CAPITULO 2
% ----------------------------------------------------------
\chapter{Pesquisa Bibliográfica}\label{cap:pesquisa}
%\addcontentsline{toc}{chapter}{Pesquisa Bibliográfica}

\minitoc
%---------------------------------------------------
% Pesquisa Bibliográfica
%---------------------------------------------------

\localtableofcontents*


Para o desenvolvimento deste trabalho, reconhecimento de acordes naturais de guitarra utilizando redes neurais artificiais, é necessário abordar alguns conceitos fundamentais e indispensáveis à sua construção, como apresentado a seguir.

\section{Formação de Acordes}

Em sua estrutura, a música possui características que podem ser descritas e analisadas através da transforma discreta de Fourier (\gls{dft})\footnote{Jean Baptiste Joseph Fourier (1768 - 1830) - Matemático e Físico Francês. Em 1807 terminou um de seus trabalhos onde observou que séries senoidais harmonicamente relacionadas eram úteis na representação da distribuição de temperatura em um corpo. Afirmou ainda que 'qualquer' sinal periódico poderia ser representado por tal série, que leva o seu nome em sua homenagem \cite{oppen2010}.}, num dos tópicos que será tratado a seguir. A relação entre as notas que compõem um acorde são semelhantes à análise feita pelas técnicas de Fourier.




% ----------------------------------------------------------
% Metodologia - CAPITULO 3
% ----------------------------------------------------------
\chapter[Método]{Método}\label{cap:metodo}
%\addcontentsline{toc}{chapter}{Método}
\minitoc
%---------------------------------------------------
% Metodologia
%---------------------------------------------------

\localtableofcontents* 


\section{Requisitos}
%
Para a guitarra que foi utilizada no processo de amostragem dos acordes, contendo 22 trastes, as frequências máxima e mínima significativas, em termo de amplitude, foram apresentadas na \autoref{tab:freqGuitarra}.

A partir dessa informação é definida a mínima $f_s$ que garanta a reconstrução do sinal. Também foram definidas as componentes de frequência mínima e máxima para o processo de reconhecimento, foram selecionadas as componente imediatamente inferior ao $E_1$, $D\#_1$ e a componente imediatamente superior ao $G_5$, $G\#_5$, totalizando 54 componentes, que serão as características que o classificador receberá como entradas. Como o objetivo é identificar o acorde como maior ou menor, o único parâmetro de entrada de cada componente será a sua amplitude.


% ----------------------------------------------------------
% Resultados Obtidos - CAPITULO 4
% ----------------------------------------------------------
\chapter[Resultados Obtidos]{Resultados Obtidos}\label{cap:result}
%\addcontentsline{toc}{chapter}{Resultados Obtidos}
\minitoc
% ----------------------------------------------------------
% Resultados Obtidos - CAPITULO 4
% ----------------------------------------------------------
\localtableofcontents* 

\section{Análise dos Resultados}

Os resultados obtidos nos ensaios para a determinação da configuração de RNA que melhor atendesse ao problema foram:


% ----------------------------------------------------------
% Conclusão - CAPITULO 5 - FINAL
% ----------------------------------------------------------
\chapter[Conclusão]{Conclusão}\label{cap:conclusao}
%\addcontentsline{toc}{chapter}{Conclusão}
%\minitoc
%%% Conclusão
%
%
\section{Conclusão}

lksdlklskmfslfmsdlf



%% ----------------------------------------------------------
%% ELEMENTOS PÓS-TEXTUAIS
%% ----------------------------------------------------------
\postextual
%% ----------------------------------------------------------
%
%% ----------------------------------------------------------
% Referências bibliográficas
%% ----------------------------------------------------------
%%\bibliography{abntex2-modelo-references}
\bibliography{Cap_06_Bibliografia}
%
%% ----------------------------------------------------------
%% Glossário
%% ----------------------------------------------------------
%%
%% Consulte o manual da classe abntex2 para orientações sobre o glossário.
%%
%%\glossary
%
%% ----------------------------------------------------------
%% Apêndices
%% ----------------------------------------------------------
%
%% ---
%% Inicia os apêndices
%% ---
%% Apêndices
%%
\begin{apendicesenv}
%
%% Imprime uma página indicando o início dos apêndices
\partapendices
%
%% ----------------------------------------------------------
\chapter{Trabalhos Publicados}\label{chap:apendice1}
%% ----------------------------------------------------------
%
\section{Artigos Publicados em Anais de Congressos e Simpósios}
\begin{enumerate}
	\item SANTOS, M. S.; SIMAS FILHO, E. F.; FARIAS, P. C. A. M; SEIXAS, J. M. Máquinas de aprendizado extremo para classificação online de eventos no
	detector ATLAS. In: \textit{XXXV Simpósio Brasileiro de Telecomunicaçõees e Processamento de Sinais (SBrT
	2017) (SBrT 2017)}. São Pedro, Brazil: [s.n.], 2017. p. 413–417
    \begin{itemize}
    	\item \textbf{Resumo}
    \end{itemize}
	
	O ATLAS é um dos detectores do LHC (\textit{Large Hadron Collider}), e está localizado no CERN (Organização Européia para a pesquisa Nuclear). Para adequada caracterização das partículas é preciso realizar uma precisa medição do perfil de deposição de energia à medida que ocorrem interações com o detector. No ATLAS os calorímetros são responsáveis por realizar a estimação da energia das partículas e, neste sentido, utilizam mais de 100.000 sensores. Um dos discriminadores para a detecção \textit{online} de elétrons utilizados no ATLAS é o \textit{Neural Ringer}, no qual o perfil de deposição de energia é utilizado como entrada para um classificador neural tipo \emph{perceptron} de múltiplas camadas. Este trabalho propõe o uso de Máquinas de Aprendizado Extremo (ELM) em substituição às redes do tipo \textit{perceptron multilayer} no \textit{Neural Ringer}. Os resultados obtidos de uma base de dados simulados apontam para uma significativa redução do tempo de treinamento, com desempenho de classificação semelhante.
	
	\item SANTOS, M. S dos; SIMAS FLHO E. F de; FARIAS, P. C. A. M; SEIXAS, J. M. Máquinas de Aprendizado Extremo e Redes com Estados de Eco para Classificação \textit{Online} de Eventos no detector ATLAS. In: \textit{Anais Do XXII Congresso Brasileiro de Automática}. 09 a 12 de setembro. João Pessoa, Brasil[S.l.]: CBA - Congresso Brasileiro De
	Automática, 2018.
	\begin{itemize}
		\item \textbf{Resumo}
	\end{itemize}
	O ATLAS é um dos detectores do acelerador de partículas LHC e com sua estrutura cilíndrica que compreende diversas camadas de sensores é capaz de caracterizar os fenômenos de interesse que ocorrem após as colisões dos feixes de partículas. O sistema de medição de energia (calorímetro) do ATLAS é composto por mais de 100.000 sensores e fornece informações importantes para a seleção \textit{online} dos eventos de interesse. Neste contexto, o \textit{Neural Ringer} é um discriminador de partículas eletromagnéticas (elétrons e fótons) que opera no sistema \textit{online} de filtragem (\textit{trigger}) do ATLAS e utiliza uma rede neural tipo Perceptron de múltiplas camadas (MLP - \textit{Multi-layer Perceptron}) para realizar a classificação das partículas a partir do perfil de deposição de energia medido nos calorímetros e formatado em anéis. Neste trabalho é proposta a substituição dos classificadores MLP do \textit{Neural Ringer} por máquinas de aprendizado com reduzido custo computacional de treinamento. São utilizadas máquinas de aprendizado extremo (ELM - \textit{Extreme Learning Machines}) e redes com estados de eco (ESN - \textit{Echo State Networks}) e resultados apontam que é possível obter eficiência de classificação semelhante ao sistema original com uma considerável redução do tempo de treinamento.
\end{enumerate}

\section{Resumo Publicado em Encontro}
\begin{enumerate}
	\item SANTOS, M. S.; SOUZA, E. E. P.; SIMAS FILHO, E. F.; FARIAS, P. C. A. M; SEIXAS, J. M; ANDRADE FILHO, L. M. Uso de Algoritmos de Treinamento Rápido para o Discriminador \textit{Neural Ringer} no Detector ATLAS. In: \textit{XXXIX Encontro Nacional de Física de Partículas e Campos (SBF - Sociedade Brasileira de Física
	2018) (SBF 2018)}. 24 a 28 de setembro, Campos do Jordão, Brazil: [s.n.], 2018.
	\begin{itemize}
		\item \textbf{Resumo}
	\end{itemize}
	O Neural Ringer é um dos algoritmos atualmente utilizados para 	identificação de elétrons no segundo nível de filtragem online do	detector ATLAS. Para prover a decisão de aceitação ou rejeição dos 	eventos, o Neural Ringer realiza um ordenamento topológico em forma de	anéis concêntricos do perfil de deposição de energia medido nos calorímetros. Neste discriminador, uma rede neural tipo \textit{perceptron} de múltiplas camadas é utilizada para classificação. Neste trabalho é	proposta a utilização de outros modelos de rede neural de treinamento rápido para realizar a etapa de classificação no Neural Ringer. Foram testados a Máquina de Aprendizado Extremo (ELM - \textit{Extreme Learning	Machine}) e a Rede de Estado de Eco (ESN - \textit{Echo State Network}). Utilizando dados simulados foi possível observar que os modelos de 	treinamento propostos obtiveram resultados de eficiência de	classificação semelhantes à versão tradicional do Neural Ringer, porém 	num tempo de treinamento consideravelmente reduzido.
\end{enumerate}
%% ====================================
\chapter{Análise de Desempenho da ELM}\label{chap:apendice2}
\section{Sensibilidade dos pesos à distribuição de probabilidade utilizada.}


As redes ELM propostas inicialmente por \citeonline{huang2004} são redes que não possuem em seu algoritmo de treino uma etapa de retropropagação do erro. Ou seja, não possui realimentação baseada no erro cometido pelo processo de treino. Sua base é a determinação da matriz $\mathbf{H}$, que representa os pesos da camada oculta da rede, expressa na~\autoref{eq:slfn2} em sua forma matricial.


\begin{eqnarray}
\vec{y}_j = \sum_{i=1}^{N} \beta_i \Phi \mathrm{(\vec{w}_i\vec{x}_j + b_i)}, \: j \in [1,M]\label{eq:slfn2}
\end{eqnarray}

A equação~\ref{eq:slfn2} pode ser reescrita como $\mathbf{H}\boldsymbol{\upbeta} = \mathbf{Y}$, sendo,
\begin{small}
	\begin{eqnarray}
	\mathbf{H} =
	\left( \begin{array}{ccc}
	\Phi(\mathrm{\vec{w}_1}\vec{x}_1 + b_1) & \ldots & \Phi(\mathrm{\vec{w}_N}\vec{x}_1 + b_N) \\
	\vdots      & \ddots & \vdots \\
	\Phi(\mathrm{\vec{w}_1}\vec{x}_M + b_1) & \ldots & \Phi(\mathrm{\vec{w}_N}\vec{x}_M + b_N)
	\end{array} \right), \label{eq:slfn_mat2}
	\end{eqnarray}
\end{small}
e $\boldsymbol{\upbeta} = (\beta^T \ldots \beta^T_N)^T$ e $Y = (y^T_1 \ldots y^T_M)^T$.

Nos trabalhos de \citeonline{huang2006}, \citeonline{huang2011} e \citeonline{huang2015} são exibidos os teoremas que dão suporte e fundamentação matemática à técnica. Alguns teoremas apresentados e demonstrados são: sua capacidade de aproximador universal, capacidade de aprendizagem e  convergência. E uma característica interessante é a forma que os pesos da camada oculta são gerados, os quais são gerados por uma função que produza números pseudo-aleatórios, e a função de ativação seja diferenciável continuamente, para que seja possível determinar os valores da matriz  $\mathbf{H}$.

Neste trabalho foi feita uma avaliação da influência da característica dos número pseudo aleatórios utilizados na camada oculta. Pois, existem diferentes tipos de funções de distribuição de probabilidade utilizadas para produção de números pseudo-aleatórios. Três formas foram avaliadas: Número gerados com distribuição normal (N1), Números com distribuição uniforme (N2) e distribuição uniforme de números inteiros pseudo-aleatórios normalizados (N3) pelo maior módulo.

Os resultados foram obtidos utilizando uma das bases de dados disponíveis, a qual é segmentada em 16 regiões, ver~\autoref{tab:segmentacaoMC2014}, (E$_T$, $|\eta|$) e são apresentados na~\autoref{tab:testELM}. Primeiro as redes foram treinadas e variando-se o número de neurônios na camada oculta até 100. em seguida, o número de neurônios que apresentou o maior índice SP dentro desse intervalo, foi utilizado para o teste de sensibilidade. Observa-se que os resultados obtidos com os número produzidos com distribuição uniforme (N2) resultaram nos menores índices SP em todas as regiões da base de teste.

\begin{table}[H]
	%\rowcolors{2}{gray!25}{white}
	\centering
	\begin{footnotesize}
	\caption{Segmentação base de dados utilizada.}
	\label{tab:segmentacaoMC2014}
	%  \resizebox{\linewidth}{!}{% Resize table to fit within \linewidth horizontally
	\setlength{\extrarowheight}{4pt}       %%Aumentar a altura das linhas
	\begin{tabular}{c*{4}c} \toprule
		\multicolumn{5}{c}{\bfseries Intervalos} \\ \midrule
		%\backslashbox{x}{y} &       0    &      1         &       2     &         3 \\
		 $E_T$ [GeV]         &  $[20;30]$ &     $[30;40]$  &   $[40;50]$ &   $[50;20.000]$ \\  \cmidrule(lr){1-1}\cmidrule(lr){2-5}
		$|\eta|$              & $[0,00;0,80]$  & $[0,80;1,37]$ & $[1,37;1,54]$ & $[1,54;2,5]$  \\ \bottomrule
	\end{tabular}
	\end{footnotesize}
\end{table}

\begin{table}[H]
	\centering
	\begin{footnotesize}
	\setlength{\extrarowheight}{2pt}
	\caption{Índices SP para três métodos de produzir números pseudo-aleatórios para a ELM.}\label{tab:testELM}
    \begin{tabular}{*{5}{c}}\toprule
    	\multicolumn{5}{c}{Índices SP para cada região da base de teste } \\ \toprule
    	&         (0,0)      &       (0,1)        &         (0,2)      &       (0,3)        \\ \cmidrule(lr){2-2}\cmidrule(lr){3-3}\cmidrule(lr){4-4}\cmidrule(lr){5-5}
	N1	& 96,366 $\pm$ 0,590 & 94,911 $\pm$ 0,963 & 94,959 $\pm$ 2,262 & 91,918 $\pm$ 0,566 \\
	N2	& 94,003 $\pm$ 1,190 & 92,370 $\pm$ 0,989 & 93,376 $\pm$ 2,587 & 90,124 $\pm$ 1,111 \\
    N3  & 96,355 $\pm$ 0,528 & 95,169 $\pm$ 0,699 & 94,328 $\pm$ 2,378 & 92,020 $\pm$ 0,461 \\ \midrule
    	&         (1,0)      &       (1,1)        &         (1,2)      &       (1,3)        \\	\cmidrule(lr){2-2}\cmidrule(lr){3-3}\cmidrule(lr){4-4}\cmidrule(lr){5-5}
	N1	& 97,398 $\pm$ 1,024 & 95,278 $\pm$ 0,940 & 90,873 $\pm$ 3,201 & 92,197 $\pm$ 1,186 \\
	N2	& 95,708 $\pm$ 2,661 & 92,468 $\pm$ 1,814 & 89,789 $\pm$ 2,855 & 88,749 $\pm$ 2,605 \\
	N3	& 97,568 $\pm$ 0,899 & 95,681 $\pm$ 0,820 & 92,148 $\pm$ 3,455 & 91,546 $\pm$ 1,165 \\ \midrule
    	&         (2,0)      &       (2,1)        &         (2,2)      &       (2,3)        \\ \cmidrule(lr){2-2}\cmidrule(lr){3-3}\cmidrule(lr){4-4}\cmidrule(lr){5-5}
	N1	& 97,693 $\pm$ 1,058 & 96,734 $\pm$ 1,495 & 99,147 $\pm$ 4,761 & 96,348 $\pm$ 1,647 \\
	N2	& 95,938 $\pm$ 1,470 & 95,892 $\pm$ 1,877 & 96,655 $\pm$ 3,990 & 94,172 $\pm$ 3,678 \\
	N3	& 97,782 $\pm$ 0,869 & 96,230 $\pm$ 1,167 & 99,716 $\pm$ 3,805 & 95,151 $\pm$ 1,879 \\ \midrule
    	&         (3,0)      &       (3,1)        &         (3,2)      &       (3,3)        \\ \cmidrule(lr){2-2}\cmidrule(lr){3-3}\cmidrule(lr){4-4}\cmidrule(lr){5-5}
	N1	& 99,209 $\pm$ 0,256 & 98,541 $\pm$ 0,564 & 99,569 $\pm$ 1,993 & 98,081 $\pm$ 0,489 \\
	N2	& 98,757 $\pm$ 0,842 & 97,501 $\pm$ 0,775 & 98,270 $\pm$ 1,867 & 96,749 $\pm$ 1,674 \\
	N3	& 99,271 $\pm$ 0,204 & 98,652 $\pm$ 0,578 & 99,482 $\pm$ 2,025 & 98,140 $\pm$ 0,464 \\ \bottomrule
    \end{tabular}
	\end{footnotesize}
\end{table}

As menores diferenças alcançadas foram de 0,45\%, enquanto que as maiores foram de 3,44\% na comparação entre os métodos N2 e N1. Na comparação entre o método N3 e N1, a menor diferença foi de 0,34\% e a maior diferença foi de 3,21\%, em favor do método N3. Outro dado possível de observar, é a incerteza alcançada pelos métodos. Em todas as regiões o método N2 produziu resultados com incerteza superior aos métodos N1 e N3.

Os resultados obtidos indicam que apesar de a técnica ELM ser flexível quanto ao método de produção dos números pseudo-aleatórios para a camada de entrada, ela possui sensibilidade, quanto às características da distribuição utilizada. Em problemas com grande volume de dados a ser processado essa variação pode ser significativa e interferir nos resultados reduzindo o desempenho do classificador.

%% ----------------------------------------------------------
%\chapter{Nullam elementum urna vel imperdiet sodales elit ipsum pharetra ligula
%ac pretium ante justo a nulla curabitur tristique arcu eu metus}
%% ----------------------------------------------------------
%\lipsum[55-57]
%
\end{apendicesenv}
% ---\label{ap:apendice1}
%\begin{apendicesenv}
%
%% Imprime uma página indicando o início dos apêndices
%\partapendices
%
%% ----------------------------------------------------------
%\chapter{Códigos}
%% ----------------------------------------------------------
%
%\lipsum[50]
%
%% ----------------------------------------------------------
%\chapter{Nullam elementum urna vel imperdiet sodales elit ipsum pharetra ligula
%ac pretium ante justo a nulla curabitur tristique arcu eu metus}
%% ----------------------------------------------------------
%\lipsum[55-57]
%
%\end{apendicesenv}
% ---


% ----------------------------------------------------------
% Anexos
% ----------------------------------------------------------

% ---
% Inicia os anexos
% ---
%\begin{anexosenv}
\begin{anexosenv}
%% Imprime uma página indicando o início dos anexos
\partanexos

\end{anexosenv}
%
%% Imprime uma página indicando o início dos anexos
%\partanexos
%
%% ---
%\chapter{Morbi ultrices rutrum lorem.}
%% ---
%\lipsum[30]
%
%% ---
%\chapter{Cras non urna sed feugiat cum sociis natoque penatibus et magnis dis
%parturient montes nascetur ridiculus mus}
%% ---
%
%\lipsum[31]
%
%% ---
%\chapter{Fusce facilisis lacinia dui}
%% ---
%
%\lipsum[32]
%
%\end{anexosenv}
%
%%---------------------------------------------------------------------
%% INDICE REMISSIVO
%%---------------------------------------------------------------------
%\phantompart
%\printindex
%---------------------------------------------------------------------

\end{document}

