\documentclass[
	% -- opções da classe memoir --
	%draft,
	11pt,				% tamanho da fonte
	openright,			% capítulos começam em pág ímpar (insere página vazia caso preciso)
	twoside,			% para impressão em verso e anverso. Oposto a oneside
	a4paper,			% tamanho do papel. 
	% -- opções da classe abntex2 --
	%chapter=TITLE,		% títulos de capítulos convertidos em letras maiúsculas
	%section=TITLE,		% títulos de seções convertidos em letras maiúsculas
	%subsection=TITLE,	% títulos de subseções convertidos em letras maiúsculas
	%subsubsection=TITLE,% títulos de subsubseções convertidos em letras maiúsculas
	% -- opções do pacote babel --
	english,			% idioma adicional para hifenização
	french,				% idioma adicional para hifenização
	spanish,			% idioma adicional para hifenização
	brazil				% o último idioma é o principal do documento
	]{abntex2}          % estilo abnt

% ---
% Pacotes básicos 
% ---
\usepackage{lmodern}			% Usa a fonte Latin Modern			
\usepackage[T1]{fontenc}		% Selecao de codigos de fonte.
\usepackage[utf8]{inputenc}		% Codificacao do documento (conversão automática dos acentos)
\usepackage{lastpage}			% Usado pela Ficha catalográfica
\usepackage{indentfirst}		% Indenta o primeiro parágrafo de cada seção.

\usepackage{color, colortbl}	% Controle das cores
\usepackage[usenames,
            dvipsnames,
            svgnames,
            table]{xcolor}	    % Controle das cores
\usepackage{graphicx}			% Inclusão de gráficos
\usepackage{microtype} 			% para melhorias de justificação
\usepackage{multirow}           % habillitar multi linhas em tabeas
\usepackage{tabularx}
\usepackage[table]{xcolor}      % permitir colorir linhas/colunas da tabela
\usepackage{hyperref}           % referências com hiperlink
\usepackage{siunitx,booktabs}           % configuraçoes extras de tabela
\usepackage{longtable}          % para construir tabelas que iniciam em uma página e terminam na outra
\usepackage{threeparttablex}     % ambiente para permitir justificar o "caption" ao tamnaho da tabela
\usepackage{verbatim}           % 
\usepackage{mathtools,
            amsmath,
            amsbsy,
            amssymb,
            amsfonts,
            dsfont,
            slashed}   % Pacotes para auxílio em ambientes matemáticos 
\usepackage[font=footnotesize,
            labelfont=bf]
            {caption}   % config p os rótulos de tabelas e figuras e habilitar inserção de sublegendas. ex: figuras com múltiplas imagens
\usepackage[skip=0.5ex]{subcaption}
\usepackage{upgreek}            % simbolos matemáticos maiúsculos
\usepackage{float}              % obriga ao corpo flutuante ficar na posição que está no código LATEX [H]
\usepackage{pdflscape}
\usepackage{multicol}           % ambiente para configuração de ambientes multicolunas
\usepackage{slashbox}           % habiitar divisao diagonal em celulas de tabelas
%\usepackage{nonfloat}
\usepackage{caption}
\usepackage{UFBA}               % personalizações para a UFBA
\usepackage{eurosym}
\usepackage{textcomp}
\usepackage{gensymb}
\usepackage{etoc}
\usepackage{nomencl}
\usepackage{lineno}             % exibir o número da linha para revisão
\makenomenclature
\usepackage[nohints,tight]{minitoc} 
\setcounter{minitocdepth}{4}
%\usepackage[outdir=./]{epstopdf}
%\usepackage{fnpct}
%\usepackage{ftnxtra}
%\usepackage{stfloats}

%\usepackage[nomain,acronym,xindy,style=super,nogroupskip]{glossaries}
\usepackage[acronyms,style=super,nogroupskip]{glossaries} % to create list of acronymous and glossaries
\usepackage[d]{esvect}
\usepackage{forest}
\usepackage{logicpuzzle}     % to allow creating grids
%\usepackage[utopia]{mathdesign}

\renewcommand\familydefault{\sfdefault}
\let\shogiH\shipH
\let\shogiV\sumV
%% ---
%% ---
% ---
%% Catologo de fontes		
%%http://www.tug.dk/FontCatalogue/mathfonts.html
% ---


% ---
% Pacotes adicionais, usados apenas no âmbito do Modelo Canônico do abnteX2
% ---
\usepackage{lipsum}				% para geração de dummy text
% ---
\usepackage{logicpuzzle}     % to allow creating grids
% ---
% Pacotes de citações
% ---
\usepackage[brazilian,hyperpageref]{backref}	 % Paginas com as citações na bibl
\usepackage[alf,bibjustif]{abntex2cite}	% Citações padrão ABNT
%\usepackage[style = abnt, % Sistema alfabético
            % style = abnt-numeric, % Sistema numérico
            % style = abnt-ibid, % Notas de referência
            %]{biblatex}
% --- 
% CONFIGURAÇÕES DE PACOTES
% --- 

% ---
% Configurações do pacote backref
% Usado sem a opção hyperpageref de backref
\renewcommand{\backrefpagesname}{Cited on page(s):~}
% Texto padrão antes do número das páginas
\renewcommand{\backref}{}
% Define os textos da citação

\renewcommand*{\backrefalt}[4]{
	\ifcase #1 %
	No one citation on text.%
	\or
	Cited on page #2.%
	\else
	Cited #1 times on pages #2.%
	\fi}

\newenvironment{figurehere}
  {\par\medskip\noindent\minipage{\linewidth}}
  {\endminipage\par\medskip}

%\newcommand\figurehere[1]{%
%\medskip\noindent\begin{minipage}{\columnwidth}
%\centering%
%#1%
%%figure,caption, and label go here
%\end{minipage}\medskip}

%% Definicao do corpo figurehere quando a figura estiver dentro do ambiente multicols
%\newenvironment{figurehere}
%  {\def\@captype{figure}}
%  {}

\newcommand{\splitcell}[1]{%
  \begin{tabular}{@{}c@{}}\strut#1\strut\end{tabular}%
}

%% Definindo comando para estilo de representacao de vetores
\newcommand{\vect}[1]{\mathbf{#1}}

%% Definindo espaco entre equacoes nos ambientes eqnarray e equation
\setlength{\jot}{11pt}
% ---
