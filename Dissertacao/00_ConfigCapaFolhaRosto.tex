% ----------------------------------------------------------
% Informações de dados para CAPA e FOLHA DE ROSTO
% ----------------------------------------------------------
%\titulo{Development of a computational method for the treatment of the crosstalk in the liquid Argon calorimeter cells of the ATLAS experiment.}
\titulo{Computational Intelligence applied for Crosstalk Reduction in a Liquid Argon Calorimeter}
\subtitulo{}  %% Inserir subtítulo, se houver

\autor{Marton Sandes dos Santos}
\local{Salvador}
\data{2024}
%\orientador[Supervisors:]{\protect \\ \hphantom{Supervisors::} Dr. Eduardo F. de Simas Filho --  UFBA \protect \\ \hphantom{Supervisors::} Dr. Bertrand Laforge -- Sorbonne Université}

\orientador[Supervisors:]{\protect \\ \hphantom{Supervisors::} Dr. Eduardo F. de Simas Filho -- UFBA \protect \\ \hphantom{Supervisors::}  Dr. Paulo César M. de A. Farias --  UFBA \protect \\ \hphantom{Supervisors::} Dr. Bertrand Laforge -- Sorbonne Université}

%\orientador[Advisor:]{Dr. Paulo César M. de A. Farias -- UFBA}
%\coorientador[Advisor:]{Dr. Eduardo F. de Simas Filho --  UFBA}
%\coorientador[Advisor:]{Dr. Bertrand Laforge -- Sorbonne Université}
\instituicao{Federal University of Bahia -- UFBA}
\faculdade{Department of Electrical and Computer Engineering}
\programa{Postgraduate Program in Electrical and Computer Engineering}
\tipotrabalho{Ph.D. Thesis}

% O preambulo deve conter o tipo do trabalho, o objetivo, 
% o nome da instituição e a área de concentração 
\preambulo{Text of the Ph.D. thesis presented to the Postgraduate Program in Electrical and Computer Engineering at the Federal University of Bahia as one of the requirements for obtaining the degree of Doctor in Electrical Engineering.}
% ---