% ---
% Configurações de aparência do PDF final
% Cores no latex: http://latexcolor.com/

% alterando o aspecto da cor azul
%\definecolor{blue}{RGB}{41,5,195}
% definindo cor usada nos hiperlinks do documento
\definecolor{persianplum}{rgb}{0.44, 0.11, 0.11}
\definecolor{midnightblue}{rgb}{0.1, 0.1, 0.44}
% cor utilizada em celulas de tabelas
\definecolor{lightgray}{rgb}{.9 .9 .9}
\definecolor{gray2}{rgb}{.69 .69 .69}

%\definecolor{orange}{rgb}{0.9, 0.36, 0.03}
% informações do PDF
\makeatletter
\hypersetup{
     	%pagebackref=true,		pdftitle={\@title}, 
		pdfauthor={\@author},
    	pdfsubject={\imprimirpreambulo},
	    pdfcreator={LaTeX with abnTeX2},
		pdfkeywords={abnt}{latex}{abntex}{abntex2}{trabalho acadêmico}, 
		colorlinks=true,       		% false: boxed links; true: colored links
%    	linkcolor=blue,          	% color of internal links
    	citecolor=persianplum,       % color of links to bibliography
    	linkcolor=midnightblue,       % color of internal links
%    	citecolor=black,            % color of links to bibliography
    	filecolor=magenta,      	% color of file links
		urlcolor=blue,
		bookmarksdepth=4
}
\makeatother
% --- 

% --- 
% Espaçamentos entre colunas
% --- 
\setlength{\columnsep}{7mm}

% --- 
% Espaçamentos entre linhas e parágrafos 
% --- 

% O tamanho do parágrafo é dado por:
\setlength{\parindent}{1.3cm}

% Controle do espaçamento entre um parágrafo e outro:
\setlength{\parskip}{0.2cm}  % tente também \onelineskip

% ---
% compila o indice
% ---
\makeindex
% ---

% ----
% pasta pas as figuras
% \graphicspath{{subdir1/}{subdir2/}{subdir3/}...{subdirn/}}
%\graphicspath{{/media/mssandes/MSSANDES1/cernbox/Pesquisa/Figuras/}}
\graphicspath{{./Figs/}}

% ----
% Glossário 
\newglossary{symbols}{sym}{sbl}{List of Symbols}
%\usepackage{filecontents}
%\begin{filecontents}{gloss.tex}
%%%%%%%%%%%%%%%%%%%%
%% Configurações para chamada do pacote glossaries




%%%%%%%%%%%%%%%%%%%%

    %================ACRONYMS=================%    
    \newacronym{rna}{RNA}{Redes Neurais Artificiais}
    \newacronym{tfg}{TFG}{Trbalho Final de Graduação}
    \newacronym{dft}{DFT}{\textit{Discret Transform Fourier}}
    

    %================SYMBOLS=================%
    \newglossaryentry{eta}{type=symbols,name=\ensuremath{\eta},sort=p,description={Pseudorapidity}}
    \newglossaryentry{alpha}{type=symbols,name=\ensuremath{\alpha},sort=l,description={Learning rate}}
    \newglossaryentry{ec}{type=symbols,name={E$_c$},sort=e,description={Critical Energy}}
    \newglossaryentry{pt}{type=symbols,name={P$_T$},sort=pt,description={Transverse Momentum}}
    \newglossaryentry{pb}{type=symbols,name={Pb},sort=pb,description={Lead}}
    \newglossaryentry{cu}{type=symbols,name={Cu},sort=cu,description={Copper}} 
    

%\GlsSetWriteIstHook{\write\glswrite{page_precedence "rnR"}}
\makenoidxglossaries
\glsnoexpandfields