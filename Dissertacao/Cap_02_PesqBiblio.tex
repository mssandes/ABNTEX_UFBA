%---------------------------------------------------
% Pesquisa Bibliográfica
%---------------------------------------------------

\localtableofcontents*


Para o desenvolvimento deste trabalho, reconhecimento de acordes naturais de guitarra utilizando redes neurais artificiais, é necessário abordar alguns conceitos fundamentais e indispensáveis à sua construção, como apresentado a seguir.

\section{Formação de Acordes}

Em sua estrutura, a música possui características que podem ser descritas e analisadas através da transforma discreta de Fourier (\gls{dft})\footnote{Jean Baptiste Joseph Fourier (1768 - 1830) - Matemático e Físico Francês. Em 1807 terminou um de seus trabalhos onde observou que séries senoidais harmonicamente relacionadas eram úteis na representação da distribuição de temperatura em um corpo. Afirmou ainda que 'qualquer' sinal periódico poderia ser representado por tal série, que leva o seu nome em sua homenagem \cite{oppen2010}.}, num dos tópicos que será tratado a seguir. A relação entre as notas que compõem um acorde são semelhantes à análise feita pelas técnicas de Fourier.


